\section*{������� E. ������� ���� ���������� ���������������� ������}

\begin{verbatim}
	@registry.Criterion
	class NoLoss(nn.Module):
	def __init__(self):
	super(NoLoss, self).__init__()
	
	def forward(self, inputs, targets):
	return inputs
	from .model import Net, OneClassNet, MaskOneClassNet
	import math
	from typing import Optional
	
	import timm
	import torch
	from efficientnet_pytorch import EfficientNet
	from torch import nn
	from torch.nn import functional as F
	from torch.nn.parameter import Parameter
	
	
	class Flatten(nn.Module):
	def forward(self, input):
	return input.view(input.size(0), -1)
	
	
	class GlobalStdPool2d(nn.Module):
	def __init__(self, flatten=False):
	super(GlobalStdPool2d, self).__init__()
	self.flatten = flatten
	
	def forward(self, x):
	if self.flatten:
	in_size = x.size()
	return x.view((in_size[0], in_size[1], -1)).std(dim=2)
	else:
	return x.view(x.size(0), x.size(1), -1).std(-1).view(x.size(0), x.size(1), 1, 1)
	
	
	class AdaptiveConcatPool2d(nn.Module):
	def __init__(self, sz=None):
	super().__init__()
	sz = sz or (1, 1)
	self.ap = nn.AdaptiveAvgPool2d(sz)
	self.mp = nn.AdaptiveMaxPool2d(sz)
	
	def forward(self, x): return torch.cat([self.mp(x), self.ap(x)], 1)
	
	
	class MeanStdConcatPool2d(nn.Module):
	def __init__(self):
	super().__init__()
	sz = (1, 1)
	self.ap = nn.AdaptiveAvgPool2d(sz)
	self.std = GlobalStdPool2d(flatten=False)
	
	def forward(self, x):
	return torch.cat([self.std(x), self.ap(x)], 1)
	
	
	class ArcMarginProduct(nn.Module):
	r"""Implement of large margin arc distance: :
	Args:
	in_features: size of each input sample
	out_features: size of each output sample
	s: norm of input feature
	m: margin
	cos(theta + m)
	"""
	
	def __init__(self, in_features, out_features):
	super(ArcMarginProduct, self).__init__()
	self.weight = Parameter(torch.FloatTensor(out_features, in_features))
	# nn.init.xavier_uniform_(self.weight)
	self.reset_parameters()
	
	def reset_parameters(self):
	stdv = 1. / math.sqrt(self.weight.size(1))
	self.weight.data.uniform_(-stdv, stdv)
	
	def forward(self, features):
	cosine = F.linear(F.normalize(features), F.normalize(self.weight.cuda()))
	return cosine
	
	
	class FeatureExtractorHead(nn.Module):
	def __init__(self, extractor, classifier):
	super().__init__()
	
	assert extractor is not None
	assert classifier is not None
	
	self.fe = extractor
	self.cls = classifier
	
	def forward(self, x):
	features = self.fe(x)
	output = self.cls(features)
	return features, output
	
	
	def freeze(model):
	for param in model.parameters():
	param.requires_grad = False
	
	
	def unfreeze(model):
	for param in model.parameters():
	param.requires_grad = True
	
	
	def get_classifier_head(input_dim, output_dim, name="concat"):
	print('Using classification head ', name)
	fe = None
	cls = None
	if name == "autoencoder":
	pass
	if name == "avg_arcface":
	fe = nn.Sequential(
	nn.AdaptiveAvgPool2d((1, 1)),
	Flatten(),
	
	nn.BatchNorm1d(input_dim),
	nn.Linear(input_dim, 512),
	nn.LeakyReLU(),
	
	nn.BatchNorm1d(512),
	nn.Dropout(p=0.5),
	
	nn.Linear(512, 512),
	nn.BatchNorm1d(512)
	)
	cls = ArcMarginProduct(512, output_dim)
	
	if name == "arcface":
	fe = nn.Sequential(
	AdaptiveConcatPool2d((1, 1)),
	Flatten(),
	
	nn.BatchNorm1d(input_dim),
	nn.Dropout(p=0.25),
	
	nn.Linear(input_dim, 512),
	nn.LeakyReLU(),
	nn.BatchNorm1d(512),
	nn.Dropout(p=0.5),
	
	nn.Linear(512, 512),
	nn.BatchNorm1d(512)
	)
	cls = ArcMarginProduct(512, output_dim)
	
	if name == "std":
	fe = nn.Sequential(
	GlobalStdPool2d(flatten=True),
	
	nn.BatchNorm1d(input_dim),
	nn.Linear(input_dim, 512),
	nn.LeakyReLU(),
	
	nn.BatchNorm1d(512),
	nn.Dropout(0.5),
	nn.Linear(512, 512),
	nn.LeakyReLU()
	)
	cls = nn.Linear(512, output_dim)
	
	if name == "meanstd":
	fe = nn.Sequential(
	MeanStdConcatPool2d(),
	Flatten(),
	
	nn.BatchNorm1d(input_dim),
	nn.Linear(input_dim, 512),
	nn.LeakyReLU(),
	
	nn.BatchNorm1d(512),
	nn.Dropout(0.5),
	nn.Linear(512, 512),
	nn.LeakyReLU()
	)
	cls = nn.Linear(512, output_dim)
	
	if name == "concat2":
	fe = nn.Sequential(
	AdaptiveConcatPool2d((1, 1)),
	Flatten(),
	
	nn.BatchNorm1d(input_dim),
	nn.Linear(input_dim, 512),
	nn.LeakyReLU(),
	
	nn.BatchNorm1d(512),
	nn.Linear(512, 512),
	nn.LeakyReLU(),
	
	nn.BatchNorm1d(512),
	nn.Dropout(p=0.5)
	)
	cls = nn.Linear(512, output_dim)
	
	if name == "concat":
	fe = nn.Sequential(
	AdaptiveConcatPool2d((1, 1)),
	Flatten(),
	nn.BatchNorm1d(input_dim),
	nn.Linear(input_dim, 256),
	nn.LeakyReLU(),
	nn.Dropout(0.5)
	)
	cls = nn.Linear(256, output_dim)
	
	if name == "avg2":
	fe = nn.Sequential(
	nn.AdaptiveAvgPool2d((1, 1)),
	Flatten(),
	
	nn.BatchNorm1d(input_dim),
	nn.Linear(input_dim, 512),
	nn.LeakyReLU(),
	
	nn.BatchNorm1d(512),
	nn.Dropout(0.5),
	nn.Linear(512, 512),
	nn.LeakyReLU()
	)
	cls = nn.Linear(512, output_dim)
	
	if name == "avg":
	fe = nn.Sequential(
	nn.AdaptiveAvgPool2d((1, 1)),
	Flatten(),
	nn.BatchNorm1d(input_dim),
	nn.Dropout(0.5),
	nn.Linear(input_dim, 256),
	nn.LeakyReLU()
	)
	cls = nn.Linear(256, output_dim)
	
	if name == "max":
	fe = nn.Sequential(
	nn.AdaptiveMaxPool2d((1, 1)),
	Flatten(),
	nn.BatchNorm1d(input_dim),
	nn.Linear(input_dim, 256),
	nn.LeakyReLU(),
	nn.Dropout(0.5)
	)
	cls = nn.Linear(256, output_dim)
	
	if name == "simple_avg":
	fe = nn.Sequential(
	nn.AdaptiveAvgPool2d((1, 1)),
	Flatten(),
	nn.BatchNorm1d(input_dim)
	)
	cls = nn.Linear(input_dim, output_dim)
	
	if name == "one_layer_fc":
	fe = nn.Sequential(
	nn.AdaptiveAvgPool2d((1, 1)),
	Flatten(),
	)
	cls = nn.Linear(input_dim, output_dim)
	
	return FeatureExtractorHead(extractor=fe, classifier=cls)
	
	
	class Net(nn.Module):
	"""
	efficientnet-b0-224
	efficientnet-b1-240
	efficientnet-b2-260
	efficientnet-b3-300
	efficientnet-b4-380
	efficientnet-b5-456
	efficientnet-b6-528
	efficientnet-b7-600
	"""
	
	def __init__(
	self,
	num_classes: int,
	n_features: int = 2048,
	arch: str = "se_resnext50_32x4d",
	class_head_name: str = "one_layer_fc",
	init_bias: Optional[float] = None):
	super().__init__()
	
	self.arch = arch
	self.n_features = n_features
	self.init_bias = init_bias
	
	if "lukemelas" in arch:
	arch = arch.split("_")[1]
	self.body = EfficientNet.from_pretrained(arch)
	else:
	self.body = timm.create_model(arch, pretrained=True)
	
	self.classification_head = get_classifier_head(n_features, num_classes, name=class_head_name)
	
	self.fill_bias()
	
	def fill_bias(self):
	bias = self.init_bias
	if bias is not None and hasattr(self.classification_head.cls, "bias"):
	print(f"Initializing bias with {bias}")
	nn.init.constant_(self.classification_head.cls.bias, bias)
	
	def forward(self, x):
	if "lukemelas" in self.arch:
	x = self.body.extract_features(x)
	else:
	x = self.body.forward_features(x)
	
	features, classification_ohe = self.classification_head(x)
	classification_ohe_softmax = F.softmax(classification_ohe, dim=1)
	classification_logits = torch.argmax(classification_ohe_softmax, axis=1) * 1.0
	
	ret = {
		'features': features,
		'logit_ohe': torch.squeeze(classification_ohe),
		'logit_ohe_softmax': torch.squeeze(classification_ohe_softmax),
		'logit': classification_logits,
		'logit_prob_positive': 1 - torch.squeeze(classification_ohe_softmax)[:, 0]
	}
	
	return ret
	
	
	class OneClassNet(Net):
	def __init__(self, num_classes: int = 1,
	n_features: int = 2048,
	arch: str = "se_resnext50_32x4d",
	class_head_name: str = "one_layer_fc",
	init_bias: Optional[float] = None
	):
	
	assert num_classes == 1
	
	super().__init__(num_classes=num_classes,
	n_features=n_features,
	arch=arch,
	class_head_name=class_head_name,
	init_bias=init_bias)
	
	def forward(self, x):
	if "lukemelas" in self.arch:
	x = self.body.extract_features(x)
	else:
	x = self.body.forward_features(x)
	
	features, logits = self.classification_head(x)
	ret = {
		'backbone_features': x,
		'features': features,
		'logits': logits,
	}
	
	return ret
	
	def predict(self, x):
	res = self.forward(x)
	
	return res["features"], res["backbone_features"], res["logits"]
	
	
	class MaskOneClassNet(Net):
	def __init__(self, num_classes: int = 1,
	n_features: int = 2048,
	arch: str = "se_resnext50_32x4d",
	class_head_name: str = "one_layer_fc",
	init_bias: Optional[float] = None
	):
	
	assert num_classes == 1
	
	super().__init__(num_classes=num_classes,
	n_features=n_features,
	arch=arch,
	class_head_name=class_head_name,
	init_bias=init_bias)
	
	self.mask_branch = nn.Sequential(
	nn.Dropout2d(0.5),
	
	nn.Conv2d(n_features, 256, kernel_size=1),
	nn.BatchNorm2d(256),
	nn.ReLU(inplace=True),
	
	nn.Conv2d(256, 256, kernel_size=1),
	nn.BatchNorm2d(256),
	nn.ReLU(inplace=True),
	
	nn.Conv2d(256, 1, kernel_size=1),
	)
	
	def forward(self, x):
	if "lukemelas" in self.arch:
	x = self.body.extract_features(x)
	else:
	x = self.body.forward_features(x)
	
	features, logits = self.classification_head(x)
	mask = self.mask_branch(x).sigmoid()
	
	melanoma = mask * logits.view(-1, 1, 1, 1)
	melanoma = melanoma.sum(dim=(2, 3)) / (mask.sum(dim=(2, 3)) + 1e-4)
	
	ret = {
		'backbone_features': x,
		
		'features': features,
		'logits': melanoma,
		'mask': mask
	}
	
	return ret
	
	def predict(self, x):
	res = self.forward(x)
	
	return res["features"], res["backbone_features"], res["logits"]
	import math
	import torch
	from torch.optim.optimizer import Optimizer, required
	
	from catalyst.dl import registry
	
	@registry.Optimizer
	class RAdamMy(Optimizer):
	def __init__(self, params, lr=1e-3, betas=(0.9, 0.999), eps=1e-4, weight_decay=0):
	if not 0.0 <= lr:
	raise ValueError("Invalid learning rate: {}".format(lr))
	if not 0.0 <= eps:
	raise ValueError("Invalid epsilon value: {}".format(eps))
	if not 0.0 <= betas[0] < 1.0:
	raise ValueError("Invalid beta parameter at index 0: {}".format(betas[0]))
	if not 0.0 <= betas[1] < 1.0:
	raise ValueError("Invalid beta parameter at index 1: {}".format(betas[1]))
	
	defaults = dict(lr=lr, betas=betas, eps=eps, weight_decay=weight_decay)
	self.buffer = [[None, None, None] for ind in range(10)]
	super(RAdamMy, self).__init__(params, defaults)
	
	def __setstate__(self, state):
	super(RAdamMy, self).__setstate__(state)
	
	def step(self, closure=None):
	
	loss = None
	if closure is not None:
	loss = closure()
	
	for group in self.param_groups:
	
	for p in group["params"]:
	if p.grad is None:
	continue
	grad = p.grad.data.float()
	if grad.is_sparse:
	raise RuntimeError("RAdam does not support sparse gradients")
	
	p_data_fp32 = p.data.float()
	
	state = self.state[p]
	
	if len(state) == 0:
	state["step"] = 0
	state["exp_avg"] = torch.zeros_like(p_data_fp32)
	state["exp_avg_sq"] = torch.zeros_like(p_data_fp32)
	else:
	state["exp_avg"] = state["exp_avg"].type_as(p_data_fp32)
	state["exp_avg_sq"] = state["exp_avg_sq"].type_as(p_data_fp32)
	
	exp_avg, exp_avg_sq = state["exp_avg"], state["exp_avg_sq"]
	beta1, beta2 = group["betas"]
	
	exp_avg_sq.mul_(beta2).addcmul_(1 - beta2, grad, grad)
	exp_avg.mul_(beta1).add_(1 - beta1, grad)
	
	state["step"] += 1
	buffered = self.buffer[int(state["step"] % 10)]
	if state["step"] == buffered[0]:
	N_sma, step_size = buffered[1], buffered[2]
	else:
	buffered[0] = state["step"]
	beta2_t = beta2 ** state["step"]
	N_sma_max = 2 / (1 - beta2) - 1
	N_sma = N_sma_max - 2 * state["step"] * beta2_t / (1 - beta2_t)
	buffered[1] = N_sma
	
	# more conservative since it's an approximated value
	if N_sma >= 5:
	step_size = math.sqrt(
	(1 - beta2_t)
	* (N_sma - 4)
	/ (N_sma_max - 4)
	* (N_sma - 2)
	/ N_sma
	* N_sma_max
	/ (N_sma_max - 2)
	) / (1 - beta1 ** state["step"])
	else:
	step_size = 1.0 / (1 - beta1 ** state["step"])
	buffered[2] = step_size
	
	if group["weight_decay"] != 0:
	p_data_fp32.add_(-group["weight_decay"] * group["lr"], p_data_fp32)
	
	# more conservative since it's an approximated value
	if N_sma >= 5:
	denom = exp_avg_sq.sqrt().add_(group["eps"])
	p_data_fp32.addcdiv_(-step_size * group["lr"], exp_avg, denom)
	else:
	p_data_fp32.add_(-step_size * group["lr"], exp_avg)
	
	p.data.copy_(p_data_fp32)
	
	return loss
	
	
	class PlainRAdam(Optimizer):
	def __init__(self, params, lr=1e-3, betas=(0.9, 0.999), eps=1e-8, weight_decay=0):
	if not 0.0 <= lr:
	raise ValueError("Invalid learning rate: {}".format(lr))
	if not 0.0 <= eps:
	raise ValueError("Invalid epsilon value: {}".format(eps))
	if not 0.0 <= betas[0] < 1.0:
	raise ValueError("Invalid beta parameter at index 0: {}".format(betas[0]))
	if not 0.0 <= betas[1] < 1.0:
	raise ValueError("Invalid beta parameter at index 1: {}".format(betas[1]))
	
	defaults = dict(lr=lr, betas=betas, eps=eps, weight_decay=weight_decay)
	
	super(PlainRAdam, self).__init__(params, defaults)
	
	def __setstate__(self, state):
	super(PlainRAdam, self).__setstate__(state)
	
	def step(self, closure=None):
	
	loss = None
	if closure is not None:
	loss = closure()
	
	for group in self.param_groups:
	
	for p in group["params"]:
	if p.grad is None:
	continue
	grad = p.grad.data.float()
	if grad.is_sparse:
	raise RuntimeError("RAdam does not support sparse gradients")
	
	p_data_fp32 = p.data.float()
	
	state = self.state[p]
	
	if len(state) == 0:
	state["step"] = 0
	state["exp_avg"] = torch.zeros_like(p_data_fp32)
	state["exp_avg_sq"] = torch.zeros_like(p_data_fp32)
	else:
	state["exp_avg"] = state["exp_avg"].type_as(p_data_fp32)
	state["exp_avg_sq"] = state["exp_avg_sq"].type_as(p_data_fp32)
	
	exp_avg, exp_avg_sq = state["exp_avg"], state["exp_avg_sq"]
	beta1, beta2 = group["betas"]
	
	exp_avg_sq.mul_(beta2).addcmul_(1 - beta2, grad, grad)
	exp_avg.mul_(beta1).add_(1 - beta1, grad)
	
	state["step"] += 1
	beta2_t = beta2 ** state["step"]
	N_sma_max = 2 / (1 - beta2) - 1
	N_sma = N_sma_max - 2 * state["step"] * beta2_t / (1 - beta2_t)
	
	if group["weight_decay"] != 0:
	p_data_fp32.add_(-group["weight_decay"] * group["lr"], p_data_fp32)
	
	# more conservative since it's an approximated value
	if N_sma >= 5:
	step_size = (
	group["lr"]
	* math.sqrt(
	(1 - beta2_t)
	* (N_sma - 4)
	/ (N_sma_max - 4)
	* (N_sma - 2)
	/ N_sma
	* N_sma_max
	/ (N_sma_max - 2)
	)
	/ (1 - beta1 ** state["step"])
	)
	denom = exp_avg_sq.sqrt().add_(group["eps"])
	p_data_fp32.addcdiv_(-step_size, exp_avg, denom)
	else:
	step_size = group["lr"] / (1 - beta1 ** state["step"])
	p_data_fp32.add_(-step_size, exp_avg)
	
	p.data.copy_(p_data_fp32)
	
	return loss
	
	
	class AdamW(Optimizer):
	def __init__(
	self, params, lr=1e-3, betas=(0.9, 0.999), eps=1e-8, weight_decay=0, warmup=0
	):
	if not 0.0 <= lr:
	raise ValueError("Invalid learning rate: {}".format(lr))
	if not 0.0 <= eps:
	raise ValueError("Invalid epsilon value: {}".format(eps))
	if not 0.0 <= betas[0] < 1.0:
	raise ValueError("Invalid beta parameter at index 0: {}".format(betas[0]))
	if not 0.0 <= betas[1] < 1.0:
	raise ValueError("Invalid beta parameter at index 1: {}".format(betas[1]))
	
	defaults = dict(
	lr=lr, betas=betas, eps=eps, weight_decay=weight_decay, warmup=warmup
	)
	super(AdamW, self).__init__(params, defaults)
	
	def __setstate__(self, state):
	super(AdamW, self).__setstate__(state)
	
	def step(self, closure=None):
	loss = None
	if closure is not None:
	loss = closure()
	
	for group in self.param_groups:
	
	for p in group["params"]:
	if p.grad is None:
	continue
	grad = p.grad.data.float()
	if grad.is_sparse:
	raise RuntimeError(
	"Adam does not support sparse gradients, please consider SparseAdam instead"
	)
	
	p_data_fp32 = p.data.float()
	
	state = self.state[p]
	
	if len(state) == 0:
	state["step"] = 0
	state["exp_avg"] = torch.zeros_like(p_data_fp32)
	state["exp_avg_sq"] = torch.zeros_like(p_data_fp32)
	else:
	state["exp_avg"] = state["exp_avg"].type_as(p_data_fp32)
	state["exp_avg_sq"] = state["exp_avg_sq"].type_as(p_data_fp32)
	
	exp_avg, exp_avg_sq = state["exp_avg"], state["exp_avg_sq"]
	beta1, beta2 = group["betas"]
	
	state["step"] += 1
	
	exp_avg_sq.mul_(beta2).addcmul_(1 - beta2, grad, grad)
	exp_avg.mul_(beta1).add_(1 - beta1, grad)
	
	denom = exp_avg_sq.sqrt().add_(group["eps"])
	bias_correction1 = 1 - beta1 ** state["step"]
	bias_correction2 = 1 - beta2 ** state["step"]
	
	if group["warmup"] > state["step"]:
	scheduled_lr = 1e-8 + state["step"] * group["lr"] / group["warmup"]
	else:
	scheduled_lr = group["lr"]
	
	step_size = (
	scheduled_lr * math.sqrt(bias_correction2) / bias_correction1
	)
	
	if group["weight_decay"] != 0:
	p_data_fp32.add_(-group["weight_decay"] * scheduled_lr, p_data_fp32)
	
	p_data_fp32.addcdiv_(-step_size, exp_avg, denom)
	
	p.data.copy_(p_data_fp32)
	
	return lossfrom .warmup import GradualWarmupScheduler
	from torch.optim.lr_scheduler import _LRScheduler
	from torch.optim.lr_scheduler import ReduceLROnPlateau
	
	
	class GradualWarmupScheduler(_LRScheduler):
	""" Gradually warm-up(increasing) learning rate in optimizer.
	Proposed in 'Accurate, Large Minibatch SGD: Training ImageNet in 1 Hour'.
	
	Args:
	optimizer (Optimizer): Wrapped optimizer.
	multiplier: target learning rate = base lr * multiplier if multiplier > 1.0. if multiplier = 1.0, lr starts from 0 and ends up with the base_lr.
	total_epoch: target learning rate is reached at total_epoch, gradually
	after_scheduler: after target_epoch, use this scheduler(eg. ReduceLROnPlateau)
	"""
	
	def __init__(self, optimizer, multiplier, total_epoch, after_scheduler=None):
	self.multiplier = multiplier
	if self.multiplier < 1.:
	raise ValueError('multiplier should be greater thant or equal to 1.')
	self.total_epoch = total_epoch
	self.after_scheduler = after_scheduler
	self.finished = False
	super(GradualWarmupScheduler, self).__init__(optimizer)
	
	def get_lr(self):
	if self.last_epoch > self.total_epoch:
	if self.after_scheduler:
	if not self.finished:
	self.after_scheduler.base_lrs = [base_lr * self.multiplier for base_lr in self.base_lrs]
	self.finished = True
	return self.after_scheduler.get_last_lr()
	return [base_lr * self.multiplier for base_lr in self.base_lrs]
	
	if self.multiplier == 1.0:
	return [base_lr * (float(self.last_epoch) / self.total_epoch) for base_lr in self.base_lrs]
	else:
	return [base_lr * ((self.multiplier - 1.) * self.last_epoch / self.total_epoch + 1.) for base_lr in
	self.base_lrs]
	
	def step_ReduceLROnPlateau(self, metrics, epoch=None):
	if epoch is None:
	epoch = self.last_epoch + 1
	self.last_epoch = epoch if epoch != 0 else 1  # ReduceLROnPlateau is called at the end of epoch, whereas others are called at beginning
	if self.last_epoch <= self.total_epoch:
	warmup_lr = [base_lr * ((self.multiplier - 1.) * self.last_epoch / self.total_epoch + 1.) for base_lr in
	self.base_lrs]
	for param_group, lr in zip(self.optimizer.param_groups, warmup_lr):
	param_group['lr'] = lr
	else:
	if epoch is None:
	self.after_scheduler.step(metrics, None)
	else:
	self.after_scheduler.step(metrics, epoch - self.total_epoch)
	
	def step(self, epoch=None, metrics=None):
	if type(self.after_scheduler) != ReduceLROnPlateau:
	if self.finished and self.after_scheduler:
	if epoch is None:
	self.after_scheduler.step(None)
	else:
	self.after_scheduler.step(epoch - self.total_epoch)
	self._last_lr = self.after_scheduler.get_last_lr()
	else:
	return super(GradualWarmupScheduler, self).step(epoch)
	else:
	self.step_ReduceLROnPlateau(metrics, epoch)
\end{verbatim}